%-*- mode: latex; fill-column: 70 -*-
% ex: set sts=4 ts=4 sw=4 et tw=70:
%&pdflatex

\documentclass[letterpaper,landscape]{report}
\usepackage[landscape,margin=0.5cm]{geometry}
\usepackage{color}
\usepackage{flowfram}
%\usepackage{booktabs}           % for rules in tables
\usepackage{tabularx}           % for column-width tables
\usepackage[table]{xcolor}      % color control

\usepackage[colorlinks]{hyperref}

\usepackage{multicol}
\usepackage{wrapfig}
%\setlength{\columnseprule}{1pt} % for visible divider
\setlength{\columnsep}{1cm}

\newcommand{\phttp}[1]{\href{http://#1}{#1}}
\newcommand{\ie}[0]{\emph{i.e.},\ }
\newcommand{\eg}[0]{\emph{e.g.},\ }
\newcommand{\etc}[0]{\emph{etc.}}


\usepackage{graphicx}
\graphicspath{
 {../}
 {../../datalad.org/src/content/pics/}
}

\usepackage{enumitem}           % useful for control of listings
\usepackage[compact,raggedright]{titlesec}
\usepackage{comment}

\newcommand{\epigraph}[3]{\textit{#1}\linebreak \vspace{-1.5em} \begin{flushright}\hspace{5em}\ --\ #2\linebreak\small{#3} \end{flushright}}

\pagestyle{empty}
\parindent=0pt

% Attempts to change bg color of *section headings
%\definecolor{secbgcol}{rgb}{0.9, 0.85, 0.85}
%\titleformat{\section}
%{\color{red}\normalfont\Large\bfseries}{\ndsection}{1em}{}
%\titleformat{\subsection}
%{\color{red}\normalfont\large\bfseries}{\begin{flushright}\hfill\thesubsection
%  \end{flushright}}{1em}{}
%
%\usepackage{pstricks}

% To create tables within multicols
\makeatletter
\newenvironment{ndtable}
  {\def\@captype{table}}
  {}


\newcommand{\ndheading}[3]{%
\vspace{0.5em}
\begin{ndtable}%
\rowcolors[\hline]{1}{#2}{} \arrayrulecolor{#3}
\begin{tabularx}{\columnwidth}{>{\centering\arraybackslash}X}\vspace{-.5em}\normalfont\large\bfseries
  #1\vspace{0.05em}\\\end{tabularx}
\end{ndtable}
\vspace{-.5em}
}

\definecolor{secfgcol}{RGB}{215, 6, 83}
\definecolor{secbgcol}{RGB}{255, 241, 248}
\newcommand{\ndsection}[1]{\ndheading{#1}{secbgcol}{secfgcol}}
\newcommand{\ndsubsection}[1]{\ndheading{#1}{secbgcol}{secfgcol}}


\begin{document}

%%
%% DEBIAN
%%
\begin{multicols}{3}    % 3 columns

% TODO: Most probably git-annex will just get 1-2 columns on the 2nd page

\begin{center}
\noindent
\includegraphics[width=0.5\columnwidth]{borrowed/git-annex-logo}

\url{http://git-annex.branchable.com}

% \hrule
\end{center}
\vspace{-1em}

\ndsection{git-annex}

is a distributed version control system for large files developed by
Joey Hess.

\ndsubsection{git-annex features}

\columnbreak
\ndsubsection{git-annex use-cases}

\ndsubsection{git-annex 101}



\columnbreak
\ndsubsection{How to install git-annex}

\ndsubsection{How to get support}

\begin{description}[nolistsep,leftmargin=1pc,style=nextline]
%\item[Overview]

%\item[GUI]
%  Use \emph{Synaptic Package Manager}
\item[On Debian systems]
  \texttt{reportbug git-annex}
\item[Community support]
  \url{http://git-annex.branchable.com/bugs}
\item[IRC] \#git-annex at OFTC network
\end{description}


\ndsubsection{Acknowledgements}

\end{multicols}


\pagebreak
%%
%% DataLad
%%
\begin{multicols}{3}    % 3 columns

%\section*{DataLad}
\begin{center}
\includegraphics[width=0.5\columnwidth]{logo1}\\
\url{http://datalad.org}
\end{center}

\ndsection{DataLad}

aims to simplify and thus facilitate delivery and sharing of
scientific data through establishing a federated data distribution.
Although originally aiming to deliver public datasets in neuroimaging
field of neuroscience, DataLad will be easy to adopt and/or extend for
any field of endeavor.

\ndsubsection{DataLad FAQ}
\begin{description}[nolistsep,leftmargin=1pc,style=nextline]
\item[Federated?] It is impractical to distribute data through
  distribution mechanisms developed \eg\ by Linux distributions, where
  content is contained within packages available from the central
  location (or its mirrors).  DataLad will only collect, unify,
  monitor, and expose through convenient interfaces data available
  across a wide range of data providers -- data sharing initiatives,
  curated collections, etc.

\item[Distributed?] DataLad uses distributed version control
  \href{http://git-scm.com}{Git} and built on top of it
  \href{http://git-annex.branchable.com}{Git-annex} for data
  logistics.  Git-annex enables \emph{distributed} operation where
  clones of the datasets could be made available across multiple sites
  and media without loosing track of data and meta-information (such
  as versioning).

\end{description}

\def\blank{\hspace{0em}\vspace{-1em}}
%\columnbreak

\ndsubsection{Planned datasets coverage}
\begin{description}[nolistsep,leftmargin=1pc,style=nextline]
\item[\phttp{OpenfMRI.org}] curated fMRI (and EEG) datasets
\item[\phttp{HumanConnectome.org}] anatomical, functional,
  diffusion MRI data from 1,200 subjects
\item[\phttp{CRCNS.org}] curated electrophysiological and neuroimaging
  datasets
\item[\href{http://fcon\_1000.projects.nitrc.org}{INDI}] collation of various datasets and initiatives (functional
  connectome, etc)
\end{description}

\ndsubsection{Planned integration}

We will expose and interface to the datasets available from

\begin{description}[nolistsep,leftmargin=1pc,style=nextline]
\item[\href{http://xnat.org}{XNAT}] widely used imaging informatics
  platform used by \phttp{HumanConnectome.org},
  \href{http://nitrc.org/ir}{NITRC-IR},
  \href{http://openfmri.org}{OpenfMRI} and others
\item[\href{http://coins.mrn.org}{COINS}] web-based neuroimaging and
  neuropsychology software suite hosting many neuroimaging datasets
\end{description}

\vspace{0.5em} Through the joint venture with the
\href{http://neuro.debian.net}{NeuroDebian} project, DataLad will also
expose itself as a \href{http://www.debian.org}{Debian} APT
repository, making it possible to \emph{install} and \emph{upgrade}
datasets using conventional tools such as \texttt{apt} and
\texttt{aptitude}.


\ndsubsection{Principal Schema}
% TODO: figure out how to make it spread over two columns
\includegraphics[width=\linewidth]{datalad-openfmri-demo_sw.png}

\ndsubsection{How could you help \emph{both of us}}

Sharing scientific data is not yet as easy as it could and should be.
Adhering to the following guidelines could help you to avoid
unnecessary burden, thus making sharing easy and thus more rewarding,
but the motto could be \emph{Be Ready}


\begin{description}[nolistsep,leftmargin=1pc,style=nextline]
\item[Legal:] Clear up and state ahead ownership (copyright) and
  (public domain dedication) license for your dataset. Provision
  public sharing in your consent forms \textbf{BEFORE} the data
  collection begins
\item[Keep detail:] Keep original detail -- copies of acquisition
  protocols, exam cards, and the DICOMs (not only NIfTIs).

\item[Be comprehensible:] Adhere to a homogeneous files structure,
  adopt and extend if necessary some standard
  (\eg\ openfmri). Consider providing dataset descriptor
  (\eg\ \url{http://dataprotocols.org/data-packages})

\item[Prepare to be reproduced:] Analyze already pre-processed
  anonymized data (where possible)

\item[Version your data:] even close to bare origin data might be
  \emph{screwed} and require revisioning.  You could
  \begin{itemize}[nolistsep,leftmargin=1pc,style=nextline]
  \item Use \href{http://git-annex.branchable.com}{Git-annex} for your
    data
  \item Consistently version older versions with the date in the
    suffix, \eg how \href{http://www.1000genomes.org}{1000genomes}
    project does
  \item If storing data in \href{http://aws.amazon.com/s3}{AWS S3},
    turn on versioning for your bucket(s)
  \end{itemize}


\item[Think about longevity:] Deposit datasets to some public/backed
  provider (\eg\ \href{http://openfmri.org}{OpenfMRI},
  \href{http://figshare.com}{figshare})

\end{description}

\ndsubsection{How could you help \emph{DataLad}}

We are picking the development speed and hope to deliver an initial
prototype within a year.  Meanwhile we would appreciate if you

\begin{description}[nolistsep,leftmargin=1pc,style=nextline]
\item[Follow\&Share]
% replace with icons for twitter/ g+
\textit{Twitter:} \url{http://twitter.com/datalad}\\
\textit{Google+:} \url{http://plus.google.com/+DataladOrg}
\textit{Blog:} \url{http://datalad.org}
\item[Discuss]
\url{https://groups.google.com/forum/#!forum/datalad}
\item[Complain\&Suggest]
\url{http://github.com/datalad/datalad/issues}\\
We are interested in use-cases, interesting datasets, feedback on
design decisions, alpha-users, \emph{etc.}
\item[Contribute]
\url{http://github.com/datalad/datalad/pulls}
\end{description}


\ndsubsection{Acknowledgements}

DataLad project development is co-funded by the US National Science
Foundation
(\href{http://www.nsf.gov/awardsearch/showAward?AWD\_ID=1429999}{NSF
  1429999} and the German Federal Ministry of Education and Research
(BMBF 01GQ1411)

\vspace{1em}
\hspace{1em}\includegraphics[width=0.4\columnwidth]{nsf1.jpg}
\hfill
\includegraphics[width=0.4\columnwidth]{bmbf_logo.jpg}\hspace{1em}

%\columnbreak
\end{multicols}



\end{document}


%%% Local Variables:
%%% mode: latex
%%% TeX-master: t
%%% TeX-PDF-mode: t
%%% whizzy-viewers: (("-pdf" "okular") ("-dvi" "xdvi") ("-ps" "gv"))
%%% End:
